In order to provide a system for mutual authentication and the required framework for sharing secrets, we decided to make use of a PKI system. For this project we will be using a custom Certificate Authority (CA). Our CA will generate new certificates using a hierarchical approach, based on three separate tiers. Each of the devices, cards and terminals alike, will possess its own unique certificate.

At the top of the hierarchy is the root certificate which establishes the 1$^{st}$ tier. The root certificate will be valid for 30 years since it is meant to be constant throughout the lifetime of the project. This certificate will conceptually represent the government and any subsequent certificates signed by the root certificate will be considered trusted by the government. This certificate will not be used directly by any of the devices, but instead will be stored offline in secure hardware.

Signed by the root certificate will be three intermediate certificates, one for each type of device: cards, charging pumps and petrol pumps. This will be the 2$^{nd}$ tier of our CA hierarchy. These three intermediate certificates will have a shorter lifetime of 10 years. They will be used to validate devices which can be trusted within the petrol rationing system. Additionally, it will be possible to detect what type of devices are taking part during a session. Thus, terminals will only communicate with cards and cards will be able to distinguish between the type of terminals easily. Only the public key of each intermediate certificate will be available on the devices. The public key will be needed to verify signatures and validate devices. The private key will remain secret and will only be available to the parties which are responsible for issuing new cards and terminals.

The third tier of the CA will consist of certificates which will belong to individual devices. These certificates will be valid for 3 years and signed by the appropriate intermediate certificate, depending on the type of the device. Devices without a valid certificate, signed by the appropriate intermediate certificate, will not be accepted within the system. They will be rejected during the handshake between cards and terminals. Both the public and private keys of each 3$^{rd}$ tier certificate will be present on the devices. By using this approach, devices will be able to establish shared secrets using the Diffie-Hellman key exchange algorithm.