The ration card is personalised in the backend before it is issued to the user. Users charge their ration cards at charging terminals to increase the petrol balance in their card. Petrol pumps are equipped with card terminals and will only supply petrol upon presentation of a valid ration card with sufficient petrol balance.

\subsection{Card Personalisation and Issue}

\begin{enumerate}
  \item Car owners apply to the government for personalised ration cards.
  \item The government verifies that the applicant does not currently have an active card and generates a personalised card loaded with: 
	\begin{itemize}
	  \item PKI certificate identifying the car owner, signed by the Intermediate Cards Certificate Authority (CA).
	  \item The corresponding private key
	  \item Intermediate Pump CA and Intermediate Charging CA
	  \item Randomly generated secret PIN.
	\end{itemize}
  \item The card and secret PIN are delivered to the car owner through a secure physical channel.
  \item Disabled cards can be brought to personalisation centers to be unlocked.
\end{enumerate}

\subsection{Petrol Withdrawal and Card Charging}
\subsubsection{Authentication Phase}
Petrol terminals and charging terminals contain a PKI certificate identifying the terminal, signed by an Intermediate CA. Each terminal also stores its private key and all of the Intermediate CAs. These certificates are used by the smartcard and terminal to mutually authenticate each other and establish an encrypted and authenticated connection. Thereafter, the card owner enters his secret PIN into the terminal. A maximum of three consecutive wrong attempts are allowed before the card is disabled. If the PIN is correct, then the user can proceed to the petrol withdrawal or charging phase, depending on the terminal.

\subsubsection{Petrol Withdrawal Phase}

\begin{enumerate}
  \item The card owner send to the terminal its current balance. The balance is signed by the terminal with which it had completed its previous transaction.
  \item The card owner inputs the amount of petrol he would like to withdraw into the terminal. The petrol pump terminal allows the transaction if and only if some security conditions are satisfied.
  \item The pump then atomically reduces the balance on the card accordingly.
  \item The fuel is released.
  \item If the tank cannot fit all the prepaid fuel, the terminal updates the balance on the card again.
  \item The card can be released.
\end{enumerate}


\subsubsection{Charging Phase}

\begin{enumerate}
  \item Charging terminals are connected to a backend database which maintains records of each car owner and their available petrol ration.
  \item If the card hasn't been charged already that month, the charging terminal increases the petrol ration in the card by the fixed monthly allowance ( 200 liters ) and updates the backend database. 
  \item If a car owner did not charge the card for the previous month, it does not carry over to the current month, but is forfeit.
\end{enumerate}


\subsubsection{Decomissioned Card}

\begin{enumerate}
  \item Cards may be decomissioned when they are lost or are expired.
  \item Lost cards are reported to the government agency.
  \item The status of these cards are updated as revoked in the government CA database.
  \item Charging Terminals no longer accept revoked cards.
\end{enumerate}
