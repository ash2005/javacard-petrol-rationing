\subsubsection{Cards}
Cards may be misplaced or stolen. The first layer of defence against abuse by an attacker is the card PIN. Cards only allow three consecutive wrong PIN entries before the card is locked and all functionality prohibited. Locked cards can be re-activated by the legitimate card owner at government centres where human staff can physically authenticate the card owner e.g. against identity cards.

The next layer of defence is for the Government CA to revoke cards which are reported to be misplaced or stolen. This is updated in the back-end database to which charging terminals are connected. When a revoked card is presented to a charging terminal, the charging terminal will disable the card by toggling a 'REVOKE-FLAG' within the card.

However, as petrol terminals are not online, they are unable to check the revocation status by OCSP. Manually updating the certificate revocation list (CRL) within the pumps is also deemed too labour intensive against the potential benefits. A third and final layer of defence for lost cards is to limit the amount of petrol withdrawals it can make in between each visit to the charging terminal. As such, cards can only make five withdrawals (which also tallies with the log space) and the total of these withdrawals cannot exceed 250 litres (which is 125\% the monthly charging allowance of 200 litres).

\subsubsection{Terminals}
