To prevent card owners from disputing filling fuel and the collection of monthly petrol allowance, all transactions will be digitally signed by both the terminal and the card and, eventually, stored in the database. These can then be referenced in cases of disputes. As such, in addition to the certificates used in Section \ref{section:mutualauth} for mutual authentication, both cards and terminals have a second certificate used for digital signatures which adheres to the same certificate hierarchy.

Every transaction between a card and a petrol terminal is first stored in the card's logbook. This logbook has space for 6 logs. Each log includes a message (comprising of the certificate ID of the terminal, updated balance and the date of transaction) and signatures from both the petrol terminal and the card on this message.

When the card is plugged into a charging terminal, these logs are uploaded automatically into the backend database and safely stored. Whenever the charging terminal updates the card balance with its monthly allowance, a log entry will similarly be created. 

Finally, card owners are responsible for not losing the card. In case that a card is missing or stolen the balance is reduced according to the defined maximum fuel filling amount (250 litres) per month. This is an inevitable consequence of keeping the petrol pumping phase off-line.